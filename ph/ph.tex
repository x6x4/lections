\documentclass[14pt]{extarticle}
\usepackage{extsizes}
\usepackage{geometry}
\geometry{margin=0.5in}

%% for images
\usepackage{graphicx}
\graphicspath{ {images/} }

%% language support
\usepackage[T1,T2A]{fontenc}
\usepackage[utf8]{inputenc}
\usepackage[english,russian]{babel}

\usepackage{amsmath}
\usepackage{tikz}

%% hyperrefs
\usepackage{hyperref}
\hypersetup{
    colorlinks,
    citecolor=black,
    filecolor=black,
    linkcolor=black,
    urlcolor=black
}

\title{2024}
\author{Кбр кыш}
\date{00.00.2024}

\begin{document}
\maketitle
\tableofcontents

\part{Геометрическая оптика}
\section{Законы оптики}
4 закона оптических явлений: 
\begin{itemize}
    \item 1) Прямолинейность распространения света.\\
    В однородной среде свет распространяется прямолинейно. 
    Это вытекает из того, что освещение непрозрачных тел 
    источником малых размеров даёт тени с резко очерченными границами.\\\\
    \textsl{При прохождении света через очень малые отверстия наблюдается
    отклонение от прямолинейности тем больше, чем меньше отверстия.} 
    \item 2) Независимость световых лучей.\\
    При пересечении лучи не возмущают друг друга, т.е. пересечение лучей
    не мешает каждому из них распространяться независимо друг от друга.\\\\
    \textsl{Работает при не очень больших интенсивностях световых пучков.
    Современные лазеры нарушают этот закон.}\\
    \\
    При попадании света на границу раздела 2х прозрачных веществ падающий луч
разделяется на 2 - отражённый и преломлённый, и их направления определяются
соответствующими законами:
    \item 3) Отражения света\\
    Отражённый луч лежит в одной плоскости с падающим лучом и нормалью,
    восстановленной в точке падения. 
    Угол падения равен углу отражения. 
    \item 4) Преломления света\\
    Преломлённый луч лежит в одной плоскости с падающим лучом и нормалью,
    восстановленной в точке падения. Отношение синуса угла падения к синусу 
    угла преломления постоянно для данных веществ.\\\\
    \textsl{и называется относительным показателем 
    преломления 2го вещества по отношению к первому.}
\end{itemize} \
\\
Закон обращения световых лучей: $n_{12} n_{21} = 1$ \\\\
Если навстречу лучу, претерпевшему ряд отражений и преломлений, пустить
другой луч, он пройдёт по тому же пути, что и первый, но в обратном 
направлении. \\\\
(Абсолютный) показатель преломления - по отношению к вакууму. \\
Оптически более плотное вещество - вещество с большим показателем преломления.
\\
При переходе из оптически более плотной среды в менее плотную луч удаляется
от нормали.\\
При углах падения от предельного угла до $\pi/2$ преломлённый луч отсутствует
(т.е. свет во вторую среду не проникает). 
\\
Полное внутреннее отражение - интенсивность отражённого луча равна 
интенсивности падающего.
\section{Теории природы света}
\begin{itemize}
    \item 1) Корпускулярная теория (Ньютон): свет - поток частиц (корпускул),
    летящих от светящегося тела по прямолинейной траектории. 
    \item 2) Волновая теория (Гюйгенс)
\end{itemize}
\ \\
Принцип Гюйгенса-Френеля: каждая точка среды, до которой доходит световое
возбуждение, является в свою очередь центром вторичных волн. Поверхность, 
огибающая в некоторый момент эти вторичные волны, указывает направление 
распространения волны в этот момент времени.\\
Волновой фронт - ГМТ точек, колеблющихся в одной фазе.\\\\
Пусть в некоторый момент времени $t$ фронт волны занимает положение $S_1$. 
Каждую точку этого фронта можно рассматривать как источник вторичных волн.
В изотропной среде они будут представлять собой сферы радиуса $v\Delta t$.
Фронтом $S_2$ волны будет огибающая этих вторичных волн.\\\\
Если плоская волна падает на границу раздела сред, то преломлённая волна 
также будет плоской.\\\\
В изотропной среде лучи перпендикулярны волновой поверхности.\\
Только волновая теория даёт верное соотношение для скорости света в среде:
$$n_{12} =  \frac{v_1}{v_2}, \ \ n = \frac{c}{v}$$\\
Путь, по которому распространяется свет в неоднородной среде,
может быть найден с помощью принципа Ферма.\\
Принцип Ферма: свет распространяется по такому пути, для прохождения
которого ему требуется минимальное время.\\
В однородной среде свет распространяется по такому пути, оптическая длина
которого минимальна.\\
Законы отражения и преломления вытекают из принципа Ферма. 

\end{document}
