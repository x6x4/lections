\documentclass[14pt]{extarticle}
\usepackage{extsizes}
\usepackage{geometry}
\geometry{margin=0.5in}

%% for images
\usepackage{graphicx}
\graphicspath{ {images/} }

%% language support
\usepackage[T1,T2A]{fontenc}
\usepackage[utf8]{inputenc}
\usepackage[english,russian]{babel}

\usepackage{amsmath}
\usepackage{tikz}

%% hyperrefs
\usepackage{hyperref}
\hypersetup{
    colorlinks,
    citecolor=black,
    filecolor=black,
    linkcolor=black,
    urlcolor=black
}

\title{2024}
\author{Кбр кыш}
\date{00.00.2024}

\begin{document}
\maketitle
\tableofcontents

\part*{Введение}

Дифференциальные уравнения делятся на:\\
- ОДУ $f(x), f^{'}(x), \dots, f^{n}(x)$\\
- УРЧП $f(x, y, z), \frac{\partial{f}}{\partial{x}}, 
\frac{\partial{f}}{\partial{y}}, ..., \frac{\partial{f^n}}{\partial{z}}$

\part{ОДУ первого порядка}
\section{Основные понятия ОДУ}
\begin{equation} \label{eq:1}
    F(x, y, y^{'}, ..., y^{(n)}) = 0
\end{equation}
- обыкновенное дифференциальное уравнение (ОДУ). \\ $F$ - известная функция,
\\$x$ - независимая переменная,\\$y(x)$ - искомая функция.
\\\\
Порядок ОДУ (\ref{eq:1}) - наивысший порядок производной неизвестной функции
$y(x)$, входящий в уравнение.\\
Примеры:\\
1) $y^{'} + y^2 ln(x) = 1$ - первого порядка\\\\
2) $xy^{(3)} + \frac{1}{x}y^4 = 0$ - третьего порядка\\
\\
Обозначения: 
\begin{itemize}
    \item $<a, b>: (a,b), [a,b], (a, b], [a, b)$ (возможны
    $\pm \infty$ для открытого конца)
    \item $R^{m}_{x_1, x_2, \dots, x_n}$ - вещественное евклидово 
    пространство переменнных $x_1, x_2, \dots, x_n$
    \item $C(D)$ - множество функций, непрерывных в области D
    \item $C^n(D)$ - множество функций, имеющих в области D непрерывные
    производные до $n$-го порядка включительно
\end{itemize}
Опр.:\\
Пусть $D \subset R^{n+2}_{x, y, y^{'}, \dots, y^{(n)}}, \ F \in C(D)$.\\
Частное решение ОДУ (\ref{eq:1}) - функция $y = \phi(x)$:\\
1) $\phi(x) \in C^n(<a,b>)$\\
2) $(x, \phi(x), \phi^{'}(x), \dots, \phi^{(n)}(x)) \in D \ \ 
\forall x \in <a,b>$\\
3) $ F(x, \phi(x), \phi^{'}(x), \dots, \phi^{(n)}(x)) \equiv 0 \ \ 
\forall x \in <a, b> $\\
\\
Пример: $y^{''} + y = 0$\\
Решения:\\
$1) \ y = sinx \ 2) \ y = 2cosx \ 3) \ y = c_1 sinx \ 4) \ y = c_2 cosx \
5) \ y = c_1 sinx + c_2 cosx \ \ \forall c_1, c_2$  \\
ОДУ может иметь бесконечно много решений.\\
Зам.: решение ОДУ не обязательно должно быть записано в явной форме;
оно может быть записано в неявной форме $\phi(x, y) = 0$ или в 
параметрической форме 
\begin{equation*}
\begin{cases}
x = x(t)\\
y = y(t)
\end{cases}
\end{equation*}
Пример:\\ 
$2xdx + 2ydy = 0$\\
$d(x^2) + d(y^2) = 0$\\
$d(x^2 + y^2) = 0$\\
$x^2 + y^2 = c$ - неявная форма,\\
\begin{equation*}
    \begin{cases}
        x = \sqrt{c} \ cost\\
        y = \sqrt{c} \ sint  
    \end{cases} 
\end{equation*} - параметрическая форма
\ \\\\
Опр.:\\
График решения $y = f(x)$ на плоскости Oxy называется интегральной кривой 
уравнения (\ref{eq:1}).\\
Опр.:\\
(\ref{eq:1}) - уравнение, не разрешимое относительно старшей производной.\\ 
Уравнение вида 
\begin{equation} \label{eq:2}
    y^{(n)} = f(x, y, y^{'}, \dots, y^{n-1})
\end{equation}
- уравнение $n$-го порядка, разрешимое относительно старшей производной.\\
Уравнение 1-го порядка 
\begin{equation} \label{eq:3}
    y^{'} = f(x, y)
\end{equation}
- ОДУ, разрешимое относительно производной.\\
\\\\
Задано (\ref{eq:3}), \ $f(x,y)$ определена на $D \subset R^2_{x,y}$\\
Постановка задачи Коши: найти решение уравнения (\ref{eq:3}), 
удовлетворяющее начальным условиям (частное решение)
\begin{equation} \label{eq:4}
    y(x_0) = y_0
\end{equation}
$x_0, y_0$ - заданные числа, $(x_0, y_0) \in D$ - начальные данные (данные
Коши)
\\\\
TODO: геометрическая интепретация задачи Коши + примеры к ней
\\\\
Примеры:\\
1) Скорость распада радия пропорциональна его массе. 
\begin{equation*} \label{eq:*}
    \frac{dm}{dt} = -\alpha m \ \ (*)
\end{equation*}
$\alpha = const > 0$ - коэффициент распада\\
$$m = c e^{-\alpha t} : \frac{dm}{dt} = -\alpha ce^{-\alpha t} = -\alpha m$$
$$m(0) = m_0 \Rightarrow m_0 = c e^0 = c$$
$$m = m_0 e^{-\alpha t}$$
2) Гармонические колебания любой природы
$$y^{''} + \omega^2 y = 0$$
$$y = c_1 \ cos \omega x + c_2 \ sin \omega x$$
Начальные условия (положение и скорость):\\
\begin{equation*}
    \begin{cases}
        y(x_0) = y_0 \\
        y^{'}(x_0) = y^{'}_0
    \end{cases}
\end{equation*}
\begin{equation*}
    \begin{cases}
        c_1 \ cos \omega x_0 + c_2 \ sin \omega x_0 = y_0\\
        -c_1 \omega\ sin \omega x_0 + c_2 \omega \ cos \omega x_0 = y^{'}_0
    \end{cases}
\end{equation*}
Теорема Крамера: $\Delta = 
\begin{vmatrix}
    cos \omega x_0 & sin \omega x_0 \\
    - \omega sin \omega x_0 & \omega cos \omega x_0 
\end{vmatrix} = \omega (sin^2 \omega x_0 + cos^2 \omega x_0) = \omega \neq 0$
\\\\\\
Пусть в (\ref{eq:3}) $f, \frac{\partial f}{\partial y} \in C(D), 
D \subset R^2_{x, y}$\\
Опр.: Общее решение ОДУ (\ref{eq:3}) - решение вида $y = \phi(x,c)$, 
зависящее от произвольной постоянной c, из которого любое ЧР этого уравнения
может быть получено надлежащим выбором значения постоянной c
(также: совокупность всех частных решений). 
Опр.: соотношение вида 
\begin{equation} \label{eq:5}
    \Phi(x, y, c) = 0    
\end{equation} называется общим интегралом ОДУ (\ref{eq:3}), если 
функция $y$, найденная из (\ref{eq:5}), задаёт общее решение (\ref{eq:3}). 
Если в соотношении (\ref{eq:5}) c принимает конкретное значение, то 
это соотношение называется частным интегралом ОДУ (\ref{eq:3}).\\
Пример:\\
$$y^{'} = \frac{2x}{3y^2}$$
$$3y^2dy = 2xdx$$
$y^3 = x^2 + C$ - общий интеграл \\ 
при $C=1\ \ y^3 = x^2 + 1$ - частный интеграл \\
$y = \sqrt[3]{x^2 + C}$ - общее решение, $y \neq 0$ 
\section{Теорема существования и единственности} 
Т. Пусть функция $f(x,y)$ непрерывна в прямоугольнике $$\Pi = 
\{(x,y) \mid  |x-x_0| \leq a, |y - y_0| \leq b\}$$ и удовлетворяет в $\Pi$
условию Липшица по аргументу $y$
$$|f(x_1, y_1) - f(x_1, y_2)| \leq L |y_1 - y_2|, L = const$$
Тогда $\exists ! y = f(x) $, решающая задачу Коши (\ref{eq:3}), (\ref{eq:4}):
\begin{equation*}
    \begin{cases}
        y^{'} = f(x, y)\\
        y(x_0) = y_0
    \end{cases}
\end{equation*}
при $x \in [x_0 - h, x_0 + h]$, где $h = \min\{a, \frac{b}{M}\}$, 
$M = \displaystyle \max_{(x,y)\in \Pi} |f(x,y)|$.\\
\begin{tikzpicture}
    \def\xcenter{0}
    \def\ycenter{0}
    \def\a{6}
    \def\b{4}
    
    % Рисуем оси координат
    \draw[->] (-1,0) -- (\a+2,0) node[right] {$x$};
    \draw[->] (0,-1) -- (0,\b+2) node[above] {$y$};
    
    % Рисуем прямоугольник
    \draw[thick] (\xcenter + 1, \ycenter + 1) 
    rectangle (\xcenter+\a + 1, \ycenter+\b + 1);
    
    % Подписываем оси и вершины прямоугольника
    \filldraw (\xcenter+1,0) circle (1pt) 
    node[below] {$x_0-a$};
    \filldraw (\xcenter+\a+1,0) circle (1pt) 
    node[below] {$x_0+a$};
    \filldraw (\xcenter+\a/2+1,0) circle (1pt) 
    node[below] {$x_0$};
    \filldraw (\xcenter,\ycenter+1) circle (1pt) 
    node[left] {$y_0-b$};
    \filldraw (\xcenter,\ycenter+\b + 1) circle (1pt) 
    node[left] {$y_0+b$};
    \filldraw (\xcenter,\ycenter+\b/2 + 1) circle (1pt) 
    node[left] {$y_0$};
    \filldraw (\xcenter+\a/2 + 1,\ycenter+\b/2 + 1) circle (1pt) 
    node[below] {$(x_0, y_0)$};

    
\end{tikzpicture} \ \\
Зам.1: нельзя утверждать, что решение есть на $[x_0 - a, x_0 + a]$.\\
Зам.2: условие Липшица может быть заменено более грубым, но зато более
легко проверяемым условием ограничения модуля частной производной
$|\frac{\partial f}{\partial y}| \leq L $ в $\Pi$.\\
По теореме Лагранжа
$$|f(x, y_1) - f(x, y_2)| = |f^{'}_y(x,y)| \ |y_1 - y_2|$$
$$|f(x, y_1) - f(x, y_2)| \leq L |y_1 - y_2| $$
Но: к примеру, $f(x, y) = |y|$ в 0 не дифференцируема, а условие Липшица
выполняется. \\
Зам.3: Продолжимость решения зависит от задачи. 
\\\\
Дальше я не поняла, да оно мб и не понадобится. 


\end{document}
