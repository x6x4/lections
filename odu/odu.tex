\documentclass[14pt]{extarticle}
\usepackage{extsizes}
\usepackage{geometry}
\geometry{margin=0.5in}

%% for images
\usepackage{graphicx}
\graphicspath{ {images/} }

%% language support
\usepackage[T1,T2A]{fontenc}
\usepackage[utf8]{inputenc}
\usepackage[english,russian]{babel}

\usepackage{amsmath}
\usepackage{tikz}

%% hyperrefs
\usepackage{hyperref}
\hypersetup{
    colorlinks,
    citecolor=black,
    filecolor=black,
    linkcolor=black,
    urlcolor=black
}

\title{2024}
\author{Кбр кыш}
\date{00.00.2024}

\begin{document}
\maketitle
\tableofcontents

\part*{Введение}

Дифференциальные уравнения делятся на:\\
- ОДУ $f(x), f^{'}(x), \dots, f^{n}(x)$\\
- УРЧП $f(x, y, z), \frac{\partial{f}}{\partial{x}}, 
\frac{\partial{f}}{\partial{y}}, ..., \frac{\partial{f^n}}{\partial{z}}$

\part{ОДУ первого порядка}
\section{Основные понятия ОДУ}
\begin{equation} \label{eq:1}
    F(x, y, y^{'}, ..., y^{(n)}) = 0
\end{equation}
- обыкновенное дифференциальное уравнение (ОДУ). \\ $F$ - известная функция,
\\$x$ - независимая переменная,\\$y(x)$ - искомая функция.
\\\\
Порядок ОДУ (\ref{eq:1}) - наивысший порядок производной неизвестной функции
$y(x)$, входящий в уравнение.\\
Примеры:\\
1) $y^{'} + y^2 ln(x) = 1$ - первого порядка\\\\
2) $xy^{(3)} + \frac{1}{x}y^4 = 0$ - третьего порядка\\
\\
Обозначения: 
\begin{itemize}
    \item $<a, b>: (a,b), [a,b], (a, b], [a, b)$ (возможны
    $\pm \infty$ для открытого конца)
    \item $R^{m}_{x_1, x_2, \dots, x_n}$ - вещественное евклидово 
    пространство переменнных $x_1, x_2, \dots, x_n$
    \item $C(D)$ - множество функций, непрерывных в области D
    \item $C^n(D)$ - множество функций, имеющих в области D непрерывные
    производные до $n$-го порядка включительно
\end{itemize}
Опр.:\\
Пусть $D \subset R^{n+2}_{x, y, y^{'}, \dots, y^{(n)}}, \ F \in C(D)$.\\
Частное решение ОДУ (\ref{eq:1}) - функция $y = \phi(x)$:\\
1) $\phi(x) \in C^n(<a,b>)$\\
2) $(x, \phi(x), \phi^{'}(x), \dots, \phi^{(n)}(x)) \in D \ \ 
\forall x \in <a,b>$\\
3) $ F(x, \phi(x), \phi^{'}(x), \dots, \phi^{(n)}(x)) \equiv 0 \ \ 
\forall x \in <a, b> $\\
\\
Пример: $y^{''} + y = 0$\\
Решения:\\
$1) \ y = sinx \ 2) \ y = 2cosx \ 3) \ y = c_1 sinx \ 4) \ y = c_2 cosx \
5) \ y = c_1 sinx + c_2 cosx \ \ \forall c_1, c_2$  \\
ОДУ может иметь бесконечно много решений.\\
Зам.: решение ОДУ не обязательно должно быть записано в явной форме;
оно может быть записано в неявной форме $\phi(x, y) = 0$ или в 
параметрической форме 
\begin{equation*}
\begin{cases}
x = x(t)\\
y = y(t)
\end{cases}
\end{equation*}
Пример:\\ 
$2xdx + 2ydy = 0$\\
$d(x^2) + d(y^2) = 0$\\
$d(x^2 + y^2) = 0$\\
$x^2 + y^2 = c$ - неявная форма,\\
\begin{equation*}
    \begin{cases}
        x = \sqrt{c} \ cost\\
        y = \sqrt{c} \ sint  
    \end{cases} 
\end{equation*} - параметрическая форма
\ \\\\
Опр.:\\
График решения $y = f(x)$ на плоскости Oxy называется интегральной кривой 
уравнения (\ref{eq:1}).\\
Опр.:\\
(\ref{eq:1}) - уравнение, не разрешимое относительно старшей производной.\\ 
Уравнение вида 
\begin{equation} \label{eq:2}
    y^{(n)} = f(x, y, y^{'}, \dots, y^{n-1})
\end{equation}
- уравнение $n$-го порядка, разрешимое относительно старшей производной.\\
Уравнение 1-го порядка 
\begin{equation} \label{eq:3}
    y^{'} = f(x, y)
\end{equation}
- ОДУ, разрешимое относительно производной.\\
\\\\
Задано (\ref{eq:3}), \ $f(x,y)$ определена на $D \subset R^2_{x,y}$\\
Постановка задачи Коши: найти решение уравнения (\ref{eq:3}), 
удовлетворяющее начальным условиям (частное решение)
\begin{equation} \label{eq:4}
    y(x_0) = y_0
\end{equation}
$x_0, y_0$ - заданные числа, $(x_0, y_0) \in D$ - начальные данные (данные
Коши)
\\\\
TODO: геометрическая интепретация задачи Коши + примеры к ней
\\\\
Примеры:\\
1) Скорость распада радия пропорциональна его массе. 
\begin{equation*} \label{eq:*}
    \frac{dm}{dt} = -\alpha m \ \ (*)
\end{equation*}
$\alpha = const > 0$ - коэффициент распада 
\\\\ 
WORK IN PROGRESS
\end{document}
